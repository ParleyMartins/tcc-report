\begin{resumo}[Resumo]
 \begin{otherlanguage*}{portuguese}

Jogos desenvolvidos nas universidades não têm muito reconhecimento ou suporte, e usuários geralmente não têm a oportunidade de jogá-los ou dar algum \textit{feedback} aos desenvolvedores. Todo o trabalho desenvolvido numa universidade, especialmente pública, deve ser acessível a toda a sociedade, desde a concepção até a implementação, mas a maioria das pessoas nem sabe que jogos são criados nas salas de aula e que esses jogos são públicos e gratuitos. Este projeto tem como objetivo fazer uma plataforma \textit{online} que permita às pessoas baixar e jogar o que for desenvolvido nas turmas de jogos da UnB. Para ajudar os desenvolvedores, o projeto também visa criar um modelo para facilitar a distribuição dos jogos, facilitando a geração de instaladores para múltiplas plataformas. Este documento explica como esta meta foi alcançada, ao se fazer um trabalho colaborativo entre várias pessoas da Universidade de Brasília. Ele também detalha o funcionamento do modelo e da plataforma desenvolvidos, apresentando suas funcionalidades e arquivos. Conclusões são tiradas a partir dos resultados obtidos e o documento termina com algumas ideias para trabalhos futuros.

 \vspace{\onelineskip}

 \noindent
 \textbf{Palavras-chaves}: jogos. desenvolvimento. plataforma. empacotamento.
 \end{otherlanguage*}
\end{resumo}
