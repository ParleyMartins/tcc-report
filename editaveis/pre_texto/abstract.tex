\begin{resumo}[Abstract]
Games developed in the University don't have much recognition or support, and, usually, users are not able to play or give feedback about any version of any of them. Everything created in the university, especially a public one, should be accessible to the society, from conception to implementation, but most people don't even know that games are created in classes and that these games are free and public. One of the objectives of this project is to make these games available to people via an online platform where they can be downloaded and play offline. To help the game developers, the project also aims to create a template to facilitate game distribution, by making it easier to generate installers for multiple platforms. This document outlines how this task was accomplished, by doing a collaborative project between various people of the \textit{Universidade de Brasília}. It also explains how the template and the platform works, by detailing their features and corresponding files. Conclusions are taken from what has been done and the document finishes by providing some ideas for future work.

   \vspace{\onelineskip}

   \noindent
   \textbf{Key-words}: games. development. platform. packaging.
\end{resumo}
