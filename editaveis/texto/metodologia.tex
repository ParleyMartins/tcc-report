\chapter[Methodology]{Methodology}
\section[Methodology]{Methodology}

This section explains what was and will be done within the duration of the
whole project. It has the main goal of creating a platform with all the games
developed in the university's courses (related to games). The games that will
be available must have all their assets and required libraries in a single
package that runs in Linux distributions without the need of installing any other package;
they also must have a graphical installer for users without technical
knowledge.

In order to achieve this goal, the games developed in this Faculty will be cataloged and cloned to a main GitHub organization (whenever possible).
It will be created, then, a packing system for two of them, one for games
using SDL 1 and the other for SDL 2. The platform itself will be developed while all the other activities take place.

After that, a script will be generated to replicate the packing system to all
of the other games, making the necessary adjustments along the way. The games
will be deployed to the website with all of their information and available
packages.

The packaging script will be integrated and adapted to the platform, so that any student who posts a game will have the packages generated automatically.

\section[Task Division]{Task Division}

Because this is a shared work among courses, teachers and students, a special
task division will be made to accomplish the established purpose.

Professor Edson and Mr. Faria were responsible for first cataloging the existing games. They will remain as helpers in the packing system and 'clients'
for the team developing the website.

The team \textit{Plataforma de Jogos UnB} from the courses 'Software
Development Methods' and 'Management of Portfolios and Projects' is
in charge of creating the actual website with some of the features desired.

The script creation and application on the games and its integration with the
platform will be my responsibility.


% The first step to achieve this goal was to catalog the games that were made
% in this Faculty, throughout the past semesters. Professor Edson Costa and Mr.
% Faria were responsible for this task. There were 20 games since the spring
% term of 2012, when the course 'Introduction to Electronic Games' was created.
% From these twenty, 16 had their sources still available and were already
% cloned to the \textit{fgagamedev} GitHub organization. However, from these sixteen,
% only eight had licenses that allowed us to change their code. Matheus and I
% were responsible for gather the other licenses with the original authors of
% the games while Prof. Edson would work on the packaging process for one of the
% licensed games.

% The next steps of the project are



\section[Technology Choice]{Technology Choice}


\section[Schedule]{Schedule}
