\chapter{Literature Review}
\label {sec:literature_review}

\section{Packages}
\label {sec:packages}

In computer science, package can have multiple meanings, depending on the context being used. A Linux package means a bundle of files containing the required data to run an application, such as binaries and information about the package.

Most Linux distributions have their own package managers. Each expects and handle different types of files, but all of them have the common goal of making the installation easier. They download the package, resolve dependencies, copy the needed binaries and execute any post- or pre-configuration required by the system to install a package \cite{linode2017linux}. For example, Debian has \textit{dpkg}, Red Hat has \textit{rpm} and Arch Linux has \textit{pacman} as default package managers.
