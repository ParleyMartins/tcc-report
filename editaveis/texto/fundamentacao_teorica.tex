\chapter{Literature Review}
\label {sec:literature_review}

\section{Repository}
\label {sec:repository}

According to the Merriam-Webster Dictionary (\citeyear{webster2017repository}), a repository is "a place, room or container where something is deposited". A software repository is a computer, directory or server that stores all the source code for that software project. This is usually available on the Internet, but it can also be local to the developers.

Repositories are also related to the version control of the source code being produced. The definition of version control is "a system that records changes to a file or set of files over time so that you can recall specific versions later" \cite{chacon2014pro}. This allows the user to compare versions, check updates, see who introduced (or removed) an issue and when and rollback to previous versions of the system \cite{chacon2014pro}. The goal is to make it easy to return to states that were working, even after changes are made after a long time.

Modern version control systems allow developers to work on a distributed basis and to parallel their tasks, with the ability of \textit{branching} the repository. Those \textit{branches} are separated lines of development, that won't mess with the main one until they are merged \cite{chacon2014pro}. This feature lets developers create and test new changes before submitting them to the project stable line of work, without affecting the final product.

\section{Packages}
\label {sec:packages}

In computer science, package can have multiple meanings, depending on the context being used. A Linux package means a bundle of files containing the required data to run an application, such as binaries and information about the package.

Most Linux distributions have their own package managers. Each expects and handle different types of files, but all of them have the common goal of making the installation easier. They download the package, resolve dependencies, copy the needed binaries and execute any post- or pre-configuration required by the system to install a package \cite{linode2017linux}. For example, Debian has \textit{dpkg}, Red Hat has \textit{rpm} and Arch Linux has \textit{pacman} as default package managers.

Another installing method is compiling from scratch. This may be very handy if the user is more advanced or the package is not in the package manager's repository, however, in this case, the user will have to manually handle dependencies, download, compile and do everything else the manager does.

\subsection{CMake}
\label {sec:cmake}

Creating packages for multiple platforms requires a lot of time and effort, because it has to be compiled on each of the systems, with those system's libraries, binaries and architecture. In order to make this task easier, several cross platform building tools were created.

CMake was created to fulfill the need for "a powerful, cross-platform build environment for the Insight Segmentation and Registration Toolkit" \cite{cmake2017overview}. It is "a system that manages the build process in an operating system and in a compiler-independent manner" \cite{cmake2017overview}.

In a very clever way, CMake generates native compiling and configuration scripts (like makefiles for Unix and namespaces for Windows) and use them to build the package \cite{cmake2017overview}. It is also designed to be used with the native environment, unlike other building tools \cite{cmake2017overview}.

The building process is controlled by files named \textit{CMakeLists.txt}. They can have commands to generate a building environment, set needed variables, compile dependencies, link required libraries and install the project. A project that has many subdirectories, each with their own rules, can have multiple CMakeLists.txt to create the final product.


\section{Game development}
\label {sec:game_development}

