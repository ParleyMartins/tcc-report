\chapter{Conclusion}
\label{sec:conclusion}

Creating games is something supposed to be challenging and fun. But just like other types of software it requires a lot of effort to distribute them to the player, specially if a developer targets multiple platforms and Operating Sytems. Each machine has diffent configuration, different libraries and archtecture and most of the development inside the University doesn't take that into account, which may cause the final software to have unexpected errors.

This work has shown just how difficult it is to support multiple OSs, seeking to create a unique solution to easier this task. Even with the care of separting the specifics for each Operating System and package the binaries within each, this project hasn't worked the way it was expected to. It wasn't possible to test the macOS distribution, due to lack of time, and Windows just gave a lot of other problems, like issues with the local environment, executables that worked partially, run time errors (that didn't happen on Linux, for example).

With all these problems, the project served best as a learning experience, from which everyone involved should take a few lessons. The first lesson is that creating the template the way it was made wasn't the correct approach, because all of the parties tried to replicate the macOS "way of packaging" on \textit{all} the platforms, instead of focusing on how to follow each platform "rules". This might have seemed a good idea, with the self contained package, but that's not how Windows or Linux work.

Another lesson is that we should take advantage of the natural environment in each plaform, by creating files to work with Visual Studio and XCode instead of 'forcing' the way around with Bash and Make. On Windows, even with the initial idea of using Visual Studio compiler, we still had Bash scripts, and when VS didn't work, we used another Linux solution, which may be the cause of all the errors and issues that happened on that OS.

The third lesson is concerning the libraries used in the project. In order to make all the libs available for the player, it was decided to use binaries and not the source files in the template (even though the source is in it too, but just to use in extreme cases). Every binary carries some information of how it was built and that may cause problems if the computer running them doesn't have all the dependencies that were there at compiling time.

On top of that, there is also the development and maintaining of the \textit{UnB Games} website. Maintaining a free software project demands time and volunteers that actually want to work with that software. It's hard to find that
