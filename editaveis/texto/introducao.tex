\chapter*[Introduction]{Introduction}
\addcontentsline{toc}{chapter}{Introduction}

Games are known to provide several benefits to the players. It may be enjoying a good story, developing new abilities and skills, bonding with friends or just relaxing after a big rushed day \cite{sandford2005games}. Independent game developers find it harder to make their games any of these goals because it's so much harder for people to see their games.

There are some courses taught in the \textit{Universidade de Brasília} (like \textit{Introdução ao Desenvolvimento de Jogos} at the campus \textit{Darcy Ribeiro}, and \textit{Introdução aos Jogos Eletrônicos} at the campus Gama) that have the goal to teach students to develop games. Students that enroll in these classes have the opportunity to learn how to create a game from scratch. Several of these students wish to continue working on game development after their graduation.

The games developed in those courses are complete, with story and gameplay but they are never seen outside class because developers have little to no experience on publishing games and there is no public place to show them. This project was created to give visibility to these games and developers and to show the work that has and will be done in this University concerning game development.

\section*{Goals}

The main goals of this project are to create an online platform to host the games developed in the courses of this University and a template that will allow game developers to quickly distribute their games. The secondary goals are the following:

\begin{itemize}
\item allow users to download, run and distribute these games in major operating systems, like Windows, macOs, Debian and Red Hat;
\item let the students of these courses easily create the respective installers and make the packages available to the public;
\item build packages to some selected games that don't have one.
\end{itemize}

\section*{Work Structure}

This document is divided into chapters. Chapter \ref{sec:basic_concepts} explains some basic concepts for the reader. Chapter \ref{sec:methodology} gives an overview of how the goals were achieved. Chapter \ref{sec:results} shows the results of the project and the difficulties to achieve those results. Chapter \ref{sec:conclusion} concludes the project and gives directions for future work.
