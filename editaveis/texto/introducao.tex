\chapter*[Introduction]{Introduction}
\addcontentsline{toc}{chapter}{Introduction}

Games are known to provide several benefits to the players. It may be enjoying a good story, developing new abilities and skills, bonding with friends or just relaxing after a big rushed day. To achieve any of these goals, independent game developers have to struggle because it's so much harder for people to see their games.

There are some courses taught here in the \textit{Universidade de Bras\'ilia} (like \textit{Introdução ao Desenvolvimento de Jogos} at the campus \textit{Darcy Ribeiro}; and \textit{Introdução aos Jogos Eletrônicos} at the campus Gama) that have the goal to teach students to develop games. The students that take these have the opportunity to learn how to create a game from scratch. Several of these students wish to continue working on game development after their graduation.

The games developed in those courses usually have a good story and are good to play with, however they are never seen outside the courses because there's nowhere to put them after they are done. People also have the tendency to relate things that are done inside the classes to things that have no use in \textit{real life}, therefore expandable.

This project was created to give visibility to these games and developers and to show the work that has and will be done in this University concerning game development.

\section*{Goals}

The main goal of this project is to create an on-line platform to host the games developed in the courses of this University. The secondary goals are the following:

\begin{itemize}
\item allow users to download, run and distribute these games in any operating system they have;
\item let the students of these courses upload their source codes and have the respective installers and packages available for the public;
\item build packages to games that don't have one.
\end{itemize}

\section*{Work Structure}

This document is divided in chapters. Chapter \ref{sec:basic_concepts} explains some basic concepts for the reader. Chapter \ref{sec:methodology} gives an overview of the tasks to be done and how they were achieved. Chapter \ref{sec:results} shows the partial results the project had so far, as well as the issues with those results. Chapter \ref{sec:future_work} describes the next steps needed to achieve the main goal, with a brief schedule containing the estimated time to complete the tasks.
